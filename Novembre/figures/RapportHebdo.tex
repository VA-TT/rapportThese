\documentclass[10pt]{beamer}
\usepackage[utf8]{inputenc}
\usepackage{tikz, pgfplots}
\pgfplotsset{compat=1.18}
\usetikzlibrary{positioning}
\usetikzlibrary{trees}
\usetikzlibrary{shapes.geometric, arrows.meta, positioning}
\usepackage{xcolor}
\usepackage{graphicx}
\usepackage{subcaption}
\usepackage{hyperref} 
\usepackage{colortbl} % Required for \rowcolor
\usepackage{animate}
\usepackage[T1]{fontenc}
\usepackage{amsmath, amsfonts, amssymb}
\usepackage{cancel}
\usepackage[french]{babel}
\usepackage[normalem]{ulem}

% Theme và màu sắc cho beamer
\usetheme{AnnArbor}
\usecolortheme{seahorse}

% Định nghĩa và thiết lập màu sắc chung
\definecolor{mydarkblue}{RGB}{0,51,102}
\setbeamercolor{structure}{fg=mydarkblue}
\setbeamercolor{frametitle}{bg=mydarkblue, fg=white}
\setbeamercolor{title}{fg=mydarkblue}
\setbeamercolor{item}{fg=mydarkblue}
\setbeamercolor{block title}{bg=mydarkblue, fg=white}
\setbeamercolor{block body}{bg=white, fg=black}
\setbeamercolor{section in toc}{fg=mydarkblue}
\setbeamercolor{subsection in toc}{fg=mydarkblue}
\setbeamercolor{caption name}{fg=mydarkblue}
\setbeamercolor{author}{fg=mydarkblue}
\setbeamercolor{date}{fg=mydarkblue}
\setbeamercolor{institute}{fg=mydarkblue}
\setbeamertemplate{caption}[numbered]
\usepackage{multirow}
\title{Rapport Hebdo}
\author{Viet Anh Quach}
\institute{3SR}
\date{\today}

\begin{document}

% \begin{frame}
%     \titlepage
% \end{frame}

\section{Étude bibliographie}
\begin{frame}{Auteur Kostas Senetakis}
    \begin{columns}
        \begin{column}{0.5\textwidth}
            \begin{figure}[h]
                \centering
                \scalebox{0.4}{\includegraphics{Experiment.png}}
            \end{figure}
        \end{column}
        \begin{column}{0.5\textwidth}
            \begin{figure}[h]
                \centering
                \scalebox{0.4}{\includegraphics{muy=constant.png}}
            \end{figure}
        \end{column}
        \end{columns}
[Kostas Senetakis, 2017,2018]:  "Under the low normal load range applied in the study, between 1 and 5 N, we found
that the frictional force is linearly correlated with the applied normal load"
\end{frame}

\section{DEM}

% 03102025
\begin{frame}{Article Stress-strain behavior of sand at high strain rates (Mehdi Omidvar et al,2012)}
            \begin{figure}[h]
                \centering
                \scalebox{0.4}{\includegraphics{StrainRate.png}}
            \end{figure}
            "Under HSR loading, there is not enough time for strain energy
accumulation, which prohibits crushing and promotes
rolling-rearrangement resulting in a higher resistance to shear"
\end{frame}


\begin{frame}{Trouver le régime de l'état critique}
    \begin{columns}
        \begin{column}{0.5\textwidth}
            \begin{figure}
                \centering
                \scalebox{0.5}{\input{3000KineticComp_ev}}
                \caption{petit déformation}
            \end{figure}
                    $\varepsilon_v = \varepsilon_x + \varepsilon_y + \varepsilon_z$
        \end{column}
        \begin{column}{0.5\textwidth}
            \begin{figure}
                \centering
                \scalebox{0.5}{\input{epsyy_epsv.tex}}
                \caption{grande déformation}
            \end{figure}
                    $\varepsilon_v = \dfrac{det(h)}{det(h_0)} - 1$
        \end{column}
    \end{columns}
\end{frame}

\begin{frame}{Trouver le régime de l'état critique}
    \begin{columns}
        \begin{column}{0.5\textwidth}
            \begin{figure}
                \centering
                \scalebox{0.5}{\input{epsyy_phi.tex}}
                \caption{Fraction solide}
            \end{figure}
        \end{column}
        \begin{column}{0.5\textwidth}
            \begin{figure}
                \centering
                \scalebox{0.5}{% GNUPLOT: LaTeX picture with Postscript
\begingroup
  \makeatletter
  \providecommand\color[2][]{%
    \GenericError{(gnuplot) \space\space\space\@spaces}{%
      Package color not loaded in conjunction with
      terminal option `colourtext'%
    }{See the gnuplot documentation for explanation.%
    }{Either use 'blacktext' in gnuplot or load the package
      color.sty in LaTeX.}%
    \renewcommand\color[2][]{}%
  }%
  \providecommand\includegraphics[2][]{%
    \GenericError{(gnuplot) \space\space\space\@spaces}{%
      Package graphicx or graphics not loaded%
    }{See the gnuplot documentation for explanation.%
    }{The gnuplot epslatex terminal needs graphicx.sty or graphics.sty.}%
    \renewcommand\includegraphics[2][]{}%
  }%
  \providecommand\rotatebox[2]{#2}%
  \@ifundefined{ifGPcolor}{%
    \newif\ifGPcolor
    \GPcolortrue
  }{}%
  \@ifundefined{ifGPblacktext}{%
    \newif\ifGPblacktext
    \GPblacktextfalse
  }{}%
  % define a \g@addto@macro without @ in the name:
  \let\gplgaddtomacro\g@addto@macro
  % define empty templates for all commands taking text:
  \gdef\gplbacktext{}%
  \gdef\gplfronttext{}%
  \makeatother
  \ifGPblacktext
    % no textcolor at all
    \def\colorrgb#1{}%
    \def\colorgray#1{}%
  \else
    % gray or color?
    \ifGPcolor
      \def\colorrgb#1{\color[rgb]{#1}}%
      \def\colorgray#1{\color[gray]{#1}}%
      \expandafter\def\csname LTw\endcsname{\color{white}}%
      \expandafter\def\csname LTb\endcsname{\color{black}}%
      \expandafter\def\csname LTa\endcsname{\color{black}}%
      \expandafter\def\csname LT0\endcsname{\color[rgb]{1,0,0}}%
      \expandafter\def\csname LT1\endcsname{\color[rgb]{0,1,0}}%
      \expandafter\def\csname LT2\endcsname{\color[rgb]{0,0,1}}%
      \expandafter\def\csname LT3\endcsname{\color[rgb]{1,0,1}}%
      \expandafter\def\csname LT4\endcsname{\color[rgb]{0,1,1}}%
      \expandafter\def\csname LT5\endcsname{\color[rgb]{1,1,0}}%
      \expandafter\def\csname LT6\endcsname{\color[rgb]{0,0,0}}%
      \expandafter\def\csname LT7\endcsname{\color[rgb]{1,0.3,0}}%
      \expandafter\def\csname LT8\endcsname{\color[rgb]{0.5,0.5,0.5}}%
    \else
      % gray
      \def\colorrgb#1{\color{black}}%
      \def\colorgray#1{\color[gray]{#1}}%
      \expandafter\def\csname LTw\endcsname{\color{white}}%
      \expandafter\def\csname LTb\endcsname{\color{black}}%
      \expandafter\def\csname LTa\endcsname{\color{black}}%
      \expandafter\def\csname LT0\endcsname{\color{black}}%
      \expandafter\def\csname LT1\endcsname{\color{black}}%
      \expandafter\def\csname LT2\endcsname{\color{black}}%
      \expandafter\def\csname LT3\endcsname{\color{black}}%
      \expandafter\def\csname LT4\endcsname{\color{black}}%
      \expandafter\def\csname LT5\endcsname{\color{black}}%
      \expandafter\def\csname LT6\endcsname{\color{black}}%
      \expandafter\def\csname LT7\endcsname{\color{black}}%
      \expandafter\def\csname LT8\endcsname{\color{black}}%
    \fi
  \fi
    \setlength{\unitlength}{0.0500bp}%
    \ifx\gptboxheight\undefined%
      \newlength{\gptboxheight}%
      \newlength{\gptboxwidth}%
      \newsavebox{\gptboxtext}%
    \fi%
    \setlength{\fboxrule}{0.5pt}%
    \setlength{\fboxsep}{1pt}%
    \definecolor{tbcol}{rgb}{1,1,1}%
\begin{picture}(7200.00,5040.00)%
    \gplgaddtomacro\gplbacktext{%
      \csname LTb\endcsname%%
      \put(946,704){\makebox(0,0)[r]{\strut{}$0.53$}}%
      \csname LTb\endcsname%%
      \put(946,1218){\makebox(0,0)[r]{\strut{}$0.54$}}%
      \csname LTb\endcsname%%
      \put(946,1733){\makebox(0,0)[r]{\strut{}$0.55$}}%
      \csname LTb\endcsname%%
      \put(946,2247){\makebox(0,0)[r]{\strut{}$0.56$}}%
      \csname LTb\endcsname%%
      \put(946,2762){\makebox(0,0)[r]{\strut{}$0.57$}}%
      \csname LTb\endcsname%%
      \put(946,3276){\makebox(0,0)[r]{\strut{}$0.58$}}%
      \csname LTb\endcsname%%
      \put(946,3790){\makebox(0,0)[r]{\strut{}$0.59$}}%
      \csname LTb\endcsname%%
      \put(946,4305){\makebox(0,0)[r]{\strut{}$0.6$}}%
      \csname LTb\endcsname%%
      \put(946,4819){\makebox(0,0)[r]{\strut{}$0.61$}}%
      \csname LTb\endcsname%%
      \put(1253,484){\makebox(0,0){\strut{}$10^{-4}$}}%
      \csname LTb\endcsname%%
      \put(3055,484){\makebox(0,0){\strut{}$10^{-3}$}}%
      \csname LTb\endcsname%%
      \put(4858,484){\makebox(0,0){\strut{}$10^{-2}$}}%
      \csname LTb\endcsname%%
      \put(6660,484){\makebox(0,0){\strut{}$10^{-1}$}}%
      \put(1651,1733){\makebox(0,0)[l]{\strut{}$\Phi_{max} = 0.5955$}}%
      \put(1651,1321){\makebox(0,0)[l]{\strut{}$b = 0.5193$}}%
    }%
    \gplgaddtomacro\gplfronttext{%
      \csname LTb\endcsname%%
      \put(209,2761){\rotatebox{-270}{\makebox(0,0){\strut{}$\phi$}}}%
      \put(3940,154){\makebox(0,0){\strut{}$I$}}%
      \csname LTb\endcsname%%
      \put(3322,1097){\makebox(0,0)[r]{\strut{}Données}}%
      \csname LTb\endcsname%%
      \put(3322,877){\makebox(0,0)[r]{\strut{}$\Phi(I)$ régression}}%
    }%
    \gplbacktext
    \put(0,0){\includegraphics[width={360.00bp},height={252.00bp}]{./3000_phi_I_fit}}%
    \gplfronttext
  \end{picture}%
\endgroup
}
                \caption{grande déformation}
            \end{figure}
        \end{column}
    \end{columns}
\end{frame}


\section{MPM}
\begin{frame}{Stabilisation d'une colonne de sol par Mohr-Coulomb}
    \begin{columns}
        \begin{column}{0.5\textwidth}
            \begin{figure}
                \centering
                \animategraphics[loop,autoplay,controls, width=5cm]{1}{MP=12_}{0}{2}
                \caption{MP=12}
            \end{figure}
        \end{column}
        \begin{column}{0.5\textwidth}
            \begin{figure}
                \centering
                \animategraphics[loop,autoplay,controls, width=5cm]{1}{MP=12+_}{0}{2}
                \caption{augementer rigidité de condition aux limites}
            \end{figure}
        \end{column}
    \end{columns}
        \end{frame}

    \begin{frame}{Effondrement d'une colonne de sol par Mohr-Coulomb}
    \begin{columns}
        \begin{column}{0.5\textwidth}
            \begin{figure}
                \centering
                \animategraphics[loop,autoplay,controls, width=5cm]{1}{MP=1200_}{0}{2}
                \caption{MP=1200}
            \end{figure}
        \end{column}
        \begin{column}{0.5\textwidth}
            \begin{figure}
                \centering
                \animategraphics[loop,autoplay,controls, width=5cm]{1}{MPM_}{0}{13}
                \caption{MP=1200 Effondrement}
                       \end{figure}
        \end{column}
    \end{columns}
        \end{frame}

\section{MPM-DEM}
Poster pour la conférence "Powder and grains"


\end{document}
