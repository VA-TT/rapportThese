\documentclass[5p,authoryear,square]{elsarticle}
\makeatletter 
\def\ps@pprintTitle{%
 \let\@oddhead\@empty
 \let\@evenhead\@empty
 \let\@evenfoot\@oddfoot} % Supprimer le bas de page ELSEVIER
\makeatother
\usepackage[utf8]{inputenc} % En unicode
\usepackage[T1]{fontenc}
\usepackage[french]{babel}
\usepackage[babel=true]{csquotes} % permet de faire \enquote{a} (« a »)
\usepackage[fleqn]{amsmath} % pour certains signes mathématiques
\usepackage{amsthm} % Pour \begin{gather}
\usepackage{booktabs} % pour \toprule (un style de tableau)
\usepackage{multirow} % Pour colonnes multiples des tableaux
\usepackage{amssymb} % Pour \leqslant (<=, >=)
\usepackage{float}
\usepackage{subcaption}
\usepackage{xcolor}
\usepackage{cancel}
\usepackage{graphicx}
\graphicspath{{figures/}}
% \definecolor{refcolor}{HTML}{0000FF}        % blue for refs
% \definecolor{citecolor}{HTML}{FF00FF}       % violet/magenta for citations
% \definecolor{footnotecolor}{HTML}{FF00FF}   % magenta for footnotes


%\bibliographystyle{elsarticle-num}
\bibliographystyle{elsarticle-harv}

\usepackage[bookmarks=true,colorlinks=true,linkcolor = red,citecolor=red,urlcolor=red]{hyperref}
\usepackage[french,nameinlink,noabbrev]{cleveref}


\begin{document}

\begin{frontmatter}

\title{\textbf{Modélisation de l'effet dynamique d'un échantillon granulaire lâche par Méthode des Élements Discrètes}}

%% Group authors per affiliation:
\author{Viet Anh QUACH, Gaël COMBE, Vincent RICHEFEU}
\address{Laboratoire 3SR, Université Grenoble Alpes}

% \author{Encadré par Sandra Ulrich Ngueveu}
% \address{LAAS-CNRS, Toulouse}

\begin{abstract}
Le main intérêt de cet article est étudiant deux aspects d'un échantillon lâche sous la compression triaxiale: le nombre de particules et le nombre d'inertie. 
Dans la domaine de résiduel (l'état critique),l'influence des termes cinétiques semble indépendant avec le valeur de la contrainte obtenu. causant par le vitess de compression (i.e) le nombre d'inertie.
C'est un pré-étude du compotement d'un volume élémentaire représentatif (VER) pour la modélisation d'un écoulement gravitaire couplant la Méthode des Élements Discrètes et la Méthode des Points Matériels.

\end{abstract}

\begin{keyword}
DEM \sep termes dynamiques \sep nombre d'inertie \sep $\mu(I)$ rhéologie \sep échantillon lâche \sep résiduel
\end{keyword}

\end{frontmatter}
% \footnotemark{}.
% \footnotetext{}

%\linenumbers
\section{Introduction}\label{introduction}
Le $\mu(I)$ rhéologie caractérisé un aspects important dans la modélisation d'un écoulement gravitaire.  

\section{Méthodologie}\label{methode}

\subsection{DEM}\label{dem}
Originé des dynamiques moléculaires, la Méthode des Éléments Discrets (DEM) a été développée dans le but d'étudier les problèmes mécaniques liés aux matériaux granulaires et géomatériaux. Contrairement aux méthodes en milieu continu telles que MPM ou FEM, DEM modélise le matériau comme un ensemble de particules discrètes.

Dans cette approche, le milieu granulaire est représenté par un assemblage de grains interagissant individuellement. Le mouvement de chaque grain est régi par la seconde loi de Newton, formulée comme suit pour le grain $i$ :

    Ensuite, dans la perspective d’une intégration au couplage MPM$\times$DEM, une seconde étude est réalisée en augmentant la vitesse de déformation. Cela permet d’examiner l’influence du nombre l'inertie sur la réponse du matériau,
Microscopic.
\begin{figure}[h] \centering
	\includegraphics[width=2in]{figures/modele.pdf}
	\caption[]{Modele DEM 3D d'un échantillon lâche (un gaz)}
	\label{echantillon} 
\end{figure}

    Le nombre d'inertie "I" est défini à partir du ``temps d'inertie'' et du ``temps de cisaillement''.
    Le temps d'inertie caractérise le temps de déplacement d'une particule moyenne de masse $m$ et de diamètre $d$, sous la pression $P$, dans $D$ dimensions\footnote{Lecture \textit{Discrete Element Modeling} – Gaël COMBE}. Leurs expressions sont :
    \begin{align}
        	\tau_c &= \dfrac{1}{\dot{\epsilon}} = \dfrac{v}{H_0}, \\
        	\tau_i &= \sqrt{\dfrac{m}{P.d^{D-2}}}
    \end{align}

    Dans notre cas (compression triaxiale en 3D, $D=3$) on obtient :
    \begin{align}
        I &= \dfrac{\tau_i}{\tau_c}
        = \dot{\epsilon}\ \sqrt{\dfrac{m}{\sigma_0.d}}
        = \dfrac{v}{H_0}\,\sqrt{\dfrac{\tfrac{4}{3}\pi R^3 \rho}{\sigma_{0}.2R}}
    \end{align}

    Où :
    \begin{itemize}
        \item $v$ : vitesse de compression,
        \item $H_0$ : hauteur initiale de l'échantillon,
        \item $R$ : rayon moyen des particules et $\rho$ : leaur masse volumique 
        \item $\sigma_{0}$ : contrainte de confinement
    \end{itemize}

\begin{table}[htbp]
\centering
\footnotesize
\begin{tabular}{@{}llll@{}}
\toprule
\textbf{Symbole} & \textbf{Paramètre} & \textbf{Valeur} & \textbf{Unité} \\
\midrule
N & Nombre de particules & $1000 \div 3375$ &  \\
R & Rayon des particules & $3 \div 5 $& mm \\
$\rho$ & Masse volumique & 2500 & kg/m$^3$ \\
$\sigma_{xx} = \sigma_{zz}$ & Contrainte isotrope & 30 & kPa \\
$k_n$, $k_t$ & Raideur norm./tang. & $3 \times 10^6$ & N/m \\
$\kappa$ & Niveau de raideur & 6250 &  \\
$\mu$ & Coefficient de frottement & 0.5 &  \\
$d_t$ & Pas de temps & $10^{-6} \div 10^{-9}$ & s \\
$\alpha$ & Coefficient d'amortissement & 0 &  \\
I & Nombre d'inertie & $10^{-4} \div 10^{-1}$ &  \\
\bottomrule
\end{tabular}
\caption{Paramètres utilisés dans la modélisation compression triaxiale DEM}
\label{parametres_triaxiale}
\end{table}


La fraction solide caractérise la dispersion des particules solides $V_s$ dans un volume $V$.  
C'est un indice important en rhéologie de l'écoulement.  $\Phi = V_s/V$.
Similairement, l'indice de vide : $e = (1-\Phi)/\Phi$


\subsection{Rhéologie}\label{rheologie}

À l'échelle macroscopique, la rhéologie $\mu(I)$ joue un rôle crucial dans la description des écoulements granulaires. Elle établit une relation constitutive entre le tenseur des contraintes du flux et le tenseur des taux de déformation  \citep{jop2006constitutive} :
\begin{equation}
\sigma_{ij} = -P \delta_{ij} + \mu(I) P \frac{\dot{\gamma}_{ij}}{\lVert \dot{\gamma} \rVert}
\label{flowTensor}
\end{equation}
où :
\begin{equation}
\mu(I) = \mu_s + \dfrac{\mu_2 - \mu_s}{1 + \dfrac{I_0}{I}}
\label{muI}
\end{equation}

En plus, la relation 
\begin{equation}
\Phi(I) = \Phi^{\max} - bI
\label{phiI}
\end{equation}

où: $\mu_s,\ \mu_2,\ I_0,\ \Phi_{\text{max}},\ b $ sont les coefficients empiriques

Seuls deux problèmes, à notre connaissance, appartiennent à cette catégorie : le \emph{DEM} rich MPM blah blah.

Dans me programme PBC3D, 2 termes cinétiques a été ajouté  dans la formulation de calcul:
\begin{itemize}
	\item $\dot{r} = h\dot{s} + \cancel{\dot{h}s}$
	\item $\ddot{s} = h^{-1}(\ddot{r} - 2\dot{h}\dot{s} - \cancel{\ddot{h}s})$
\end{itemize}

\begin{equation}
    \sigma_p = -\frac{1}{|\det h_p|} \left( \sum_k f_k \otimes \ell_k + \sum_n m_n \dot{r}_n \otimes \dot{r}_n \right)
    \label{eq:contrainteDynamique}
\end{equation}	
the first term denotes Love-Weber stress, while the
second (kinetic) term may become significant during extremely rapid deformations

\begin{equation}
    I = \dot{\varepsilon} \times \sqrt{\frac{m}{\sigma_{33} \times \bar{a}}}
 = \sqrt{\frac{\pi}{6}} \times \frac{\dot{\varepsilon} .\bar{a}}{\sqrt{\sigma_{33}/\rho_s}}
    \label{eq:placeholder_label}
\end{equation}



\subsection{Critère de rupture de Mohr} \label{Mohr}
Dans le sable sec, autre dire sans cohésif, les interactions entre les particules sont purement frottant. 
On utilise le critère de Mohr pour évaluer le compotment à l'échelle microscopic à l'echelle macroscopic.
Dans le cas simplicité, l'enveloppe de Mohr est simplifié par un droit transversale tous les cercles de Mohr.
La résistance au force de cissailement du matériau est composant de deux indices: le valeur $\mu = \tan(\phi)$, calculé selon l'angle de frottement $\phi$ de la droit et la cohésion apparente $C$, qui est est la intersection entre la droit avec l'axe y (i.e l'axe de cissailement)
Un point particulier de la méthode DEM est que l’on agit uniquement sur les paramètres microscopiques, alors que les comportements à grande échelle émergent naturellement.  
Cette propriété peut être vérifiée en comparant avec les comportements macroscopiques bien connus en mécanique des sols.  

Le cercle de Mohr est une méthode bien connue pour identifier la résistance au cisaillement maximale du sol.  
À partir de ses courbes, on peut déterminer la cohésion $c$ et l’angle de frottement interne $\varphi$ selon la relation :
\begin{equation}
    \tau = \sigma_n \tan(\varphi) + c
    \label{eq:tangentMohr}
\end{equation}
La pente de la droite tangente aux cercles, soit $\tan(\varphi) = \mu$, reflète le frottement interne à l’état considéré.  
Le $\mu_{\text{résiduel}}$ correspond à la pente de la droite tangente au cercle de Mohr à l’état critique.  
On considère généralement que ce $\mu_{\text{résiduel}}$ est une valeur stationnaire.  
Dans notre cas, le matériau étudié est du sable sec. Le point d’intersection entre la droite tangente aux cercles et l’axe vertical, selon la théorie, doit être nul, ce qui correspond à une cohésion nulle ($c=0$).

\section{Résultat et discussion}\label{resultat}

\subsection{Nombre des particules}\label{N}

            \begin{figure}
                \centering
                \scalebox{0.5}{\input{figures/ComparerNP.tex}}
                \caption{Étude sur nombre de particules}
                \label{comparerNP}
            \end{figure}
\subsection{Influence de termes dynamiques ajouté}\label{termeDynamique}
Sauf au pic, la contrainte dans la régime résiduel n'affect pas la contrainte, consequently le $\mu_{\text{résiduel}}$

\begin{figure}[htbp]
    \centering
    \begin{subfigure}[b]{0.45\textwidth}
        \scalebox{0.5}{% GNUPLOT: LaTeX picture with Postscript
\begingroup
  \makeatletter
  \providecommand\color[2][]{%
    \GenericError{(gnuplot) \space\space\space\@spaces}{%
      Package color not loaded in conjunction with
      terminal option `colourtext'%
    }{See the gnuplot documentation for explanation.%
    }{Either use 'blacktext' in gnuplot or load the package
      color.sty in LaTeX.}%
    \renewcommand\color[2][]{}%
  }%
  \providecommand\includegraphics[2][]{%
    \GenericError{(gnuplot) \space\space\space\@spaces}{%
      Package graphicx or graphics not loaded%
    }{See the gnuplot documentation for explanation.%
    }{The gnuplot epslatex terminal needs graphicx.sty or graphics.sty.}%
    \renewcommand\includegraphics[2][]{}%
  }%
  \providecommand\rotatebox[2]{#2}%
  \@ifundefined{ifGPcolor}{%
    \newif\ifGPcolor
    \GPcolortrue
  }{}%
  \@ifundefined{ifGPblacktext}{%
    \newif\ifGPblacktext
    \GPblacktextfalse
  }{}%
  % define a \g@addto@macro without @ in the name:
  \let\gplgaddtomacro\g@addto@macro
  % define empty templates for all commands taking text:
  \gdef\gplbacktext{}%
  \gdef\gplfronttext{}%
  \makeatother
  \ifGPblacktext
    % no textcolor at all
    \def\colorrgb#1{}%
    \def\colorgray#1{}%
  \else
    % gray or color?
    \ifGPcolor
      \def\colorrgb#1{\color[rgb]{#1}}%
      \def\colorgray#1{\color[gray]{#1}}%
      \expandafter\def\csname LTw\endcsname{\color{white}}%
      \expandafter\def\csname LTb\endcsname{\color{black}}%
      \expandafter\def\csname LTa\endcsname{\color{black}}%
      \expandafter\def\csname LT0\endcsname{\color[rgb]{1,0,0}}%
      \expandafter\def\csname LT1\endcsname{\color[rgb]{0,1,0}}%
      \expandafter\def\csname LT2\endcsname{\color[rgb]{0,0,1}}%
      \expandafter\def\csname LT3\endcsname{\color[rgb]{1,0,1}}%
      \expandafter\def\csname LT4\endcsname{\color[rgb]{0,1,1}}%
      \expandafter\def\csname LT5\endcsname{\color[rgb]{1,1,0}}%
      \expandafter\def\csname LT6\endcsname{\color[rgb]{0,0,0}}%
      \expandafter\def\csname LT7\endcsname{\color[rgb]{1,0.3,0}}%
      \expandafter\def\csname LT8\endcsname{\color[rgb]{0.5,0.5,0.5}}%
    \else
      % gray
      \def\colorrgb#1{\color{black}}%
      \def\colorgray#1{\color[gray]{#1}}%
      \expandafter\def\csname LTw\endcsname{\color{white}}%
      \expandafter\def\csname LTb\endcsname{\color{black}}%
      \expandafter\def\csname LTa\endcsname{\color{black}}%
      \expandafter\def\csname LT0\endcsname{\color{black}}%
      \expandafter\def\csname LT1\endcsname{\color{black}}%
      \expandafter\def\csname LT2\endcsname{\color{black}}%
      \expandafter\def\csname LT3\endcsname{\color{black}}%
      \expandafter\def\csname LT4\endcsname{\color{black}}%
      \expandafter\def\csname LT5\endcsname{\color{black}}%
      \expandafter\def\csname LT6\endcsname{\color{black}}%
      \expandafter\def\csname LT7\endcsname{\color{black}}%
      \expandafter\def\csname LT8\endcsname{\color{black}}%
    \fi
  \fi
    \setlength{\unitlength}{0.0500bp}%
    \ifx\gptboxheight\undefined%
      \newlength{\gptboxheight}%
      \newlength{\gptboxwidth}%
      \newsavebox{\gptboxtext}%
    \fi%
    \setlength{\fboxrule}{0.5pt}%
    \setlength{\fboxsep}{1pt}%
    \definecolor{tbcol}{rgb}{1,1,1}%
\begin{picture}(7200.00,5040.00)%
    \gplgaddtomacro\gplbacktext{%
      \csname LTb\endcsname%%
      \put(814,704){\makebox(0,0)[r]{\strut{}$0$}}%
      \put(814,1218){\makebox(0,0)[r]{\strut{}$20$}}%
      \put(814,1733){\makebox(0,0)[r]{\strut{}$40$}}%
      \put(814,2247){\makebox(0,0)[r]{\strut{}$60$}}%
      \put(814,2762){\makebox(0,0)[r]{\strut{}$80$}}%
      \put(814,3276){\makebox(0,0)[r]{\strut{}$100$}}%
      \put(814,3790){\makebox(0,0)[r]{\strut{}$120$}}%
      \put(814,4305){\makebox(0,0)[r]{\strut{}$140$}}%
      \put(814,4819){\makebox(0,0)[r]{\strut{}$160$}}%
      \put(946,484){\makebox(0,0){\strut{}$0$}}%
      \put(1922,484){\makebox(0,0){\strut{}$10$}}%
      \put(2898,484){\makebox(0,0){\strut{}$20$}}%
      \put(3875,484){\makebox(0,0){\strut{}$30$}}%
      \put(4851,484){\makebox(0,0){\strut{}$40$}}%
      \put(5827,484){\makebox(0,0){\strut{}$50$}}%
      \put(6803,484){\makebox(0,0){\strut{}$60$}}%
    }%
    \gplgaddtomacro\gplfronttext{%
      \csname LTb\endcsname%%
      \put(341,2761){\rotatebox{-270}{\makebox(0,0){\strut{}q (kPa)}}}%
      \put(3874,154){\makebox(0,0){\strut{}$\varepsilon_{yy}$ (\%)}}%
      \csname LTb\endcsname%%
      \put(5888,4636){\makebox(0,0)[r]{\strut{}$I = 10^{-3}$ cinétique}}%
      \csname LTb\endcsname%%
      \put(5888,4396){\makebox(0,0)[r]{\strut{}$I = 10^{-3}$ sans cinétique}}%
      \csname LTb\endcsname%%
      \put(5888,4156){\makebox(0,0)[r]{\strut{}$I = 10^{-2}$ cinétique}}%
      \csname LTb\endcsname%%
      \put(5888,3916){\makebox(0,0)[r]{\strut{}$I = 10^{-2}$ sans cinétique}}%
      \csname LTb\endcsname%%
      \put(5888,3676){\makebox(0,0)[r]{\strut{}$I = 10^{-1}$ cinétique}}%
      \csname LTb\endcsname%%
      \put(5888,3436){\makebox(0,0)[r]{\strut{}$I = 10^{-1}$ sans cinétique}}%
    }%
    \gplbacktext
    \put(0,0){\includegraphics[width={360.00bp},height={252.00bp}]{./3000KineticComp_q}}%
    \gplfronttext
  \end{picture}%
\endgroup
}
        \caption{Contrainte déviatorique}
        \label{fig:3000_kinetic_q}
    \end{subfigure}
    \hfill
    \begin{subfigure}[b]{0.45\textwidth}
        \scalebox{0.5}{\input{figures/3000KineticComp_ev.tex}}
        \caption{Déformation volumique}
        \label{fig:3000_kinetic_ev}
    \end{subfigure}
    \caption{Influence des termes cinétiques pour N=3000}
    \label{fig:3000_kinetic_comp}
\end{figure}

\subsection{Comportement microscopique à macroscopique}\label{figureMuy}
\begin{figure}[htbp]
    \centering
    \begin{subfigure}[b]{0.45\textwidth}
        \centering
        \scalebox{0.5}{\input{figures/1000KineticComp.tex}}
        \caption{Contrainte déviatorique compression}
        \label{fig:3000_kinetic_q}
    \end{subfigure}
        \begin{subfigure}[b]{0.45\textwidth}
                        \centering
        \scalebox{0.5}{\input{figures/1000KineticExt.tex}}
        \caption{Contrainte déviatorique extension}
        \label{fig:3000_kinetic_q}
    \end{subfigure}
    \begin{subfigure}[b]{0.45\textwidth}
                    \centering
        \scalebox{0.5}{% GNUPLOT: LaTeX picture with Postscript
\begingroup
  \makeatletter
  \providecommand\color[2][]{%
    \GenericError{(gnuplot) \space\space\space\@spaces}{%
      Package color not loaded in conjunction with
      terminal option `colourtext'%
    }{See the gnuplot documentation for explanation.%
    }{Either use 'blacktext' in gnuplot or load the package
      color.sty in LaTeX.}%
    \renewcommand\color[2][]{}%
  }%
  \providecommand\includegraphics[2][]{%
    \GenericError{(gnuplot) \space\space\space\@spaces}{%
      Package graphicx or graphics not loaded%
    }{See the gnuplot documentation for explanation.%
    }{The gnuplot epslatex terminal needs graphicx.sty or graphics.sty.}%
    \renewcommand\includegraphics[2][]{}%
  }%
  \providecommand\rotatebox[2]{#2}%
  \@ifundefined{ifGPcolor}{%
    \newif\ifGPcolor
    \GPcolortrue
  }{}%
  \@ifundefined{ifGPblacktext}{%
    \newif\ifGPblacktext
    \GPblacktextfalse
  }{}%
  % define a \g@addto@macro without @ in the name:
  \let\gplgaddtomacro\g@addto@macro
  % define empty templates for all commands taking text:
  \gdef\gplbacktext{}%
  \gdef\gplfronttext{}%
  \makeatother
  \ifGPblacktext
    % no textcolor at all
    \def\colorrgb#1{}%
    \def\colorgray#1{}%
  \else
    % gray or color?
    \ifGPcolor
      \def\colorrgb#1{\color[rgb]{#1}}%
      \def\colorgray#1{\color[gray]{#1}}%
      \expandafter\def\csname LTw\endcsname{\color{white}}%
      \expandafter\def\csname LTb\endcsname{\color{black}}%
      \expandafter\def\csname LTa\endcsname{\color{black}}%
      \expandafter\def\csname LT0\endcsname{\color[rgb]{1,0,0}}%
      \expandafter\def\csname LT1\endcsname{\color[rgb]{0,1,0}}%
      \expandafter\def\csname LT2\endcsname{\color[rgb]{0,0,1}}%
      \expandafter\def\csname LT3\endcsname{\color[rgb]{1,0,1}}%
      \expandafter\def\csname LT4\endcsname{\color[rgb]{0,1,1}}%
      \expandafter\def\csname LT5\endcsname{\color[rgb]{1,1,0}}%
      \expandafter\def\csname LT6\endcsname{\color[rgb]{0,0,0}}%
      \expandafter\def\csname LT7\endcsname{\color[rgb]{1,0.3,0}}%
      \expandafter\def\csname LT8\endcsname{\color[rgb]{0.5,0.5,0.5}}%
    \else
      % gray
      \def\colorrgb#1{\color{black}}%
      \def\colorgray#1{\color[gray]{#1}}%
      \expandafter\def\csname LTw\endcsname{\color{white}}%
      \expandafter\def\csname LTb\endcsname{\color{black}}%
      \expandafter\def\csname LTa\endcsname{\color{black}}%
      \expandafter\def\csname LT0\endcsname{\color{black}}%
      \expandafter\def\csname LT1\endcsname{\color{black}}%
      \expandafter\def\csname LT2\endcsname{\color{black}}%
      \expandafter\def\csname LT3\endcsname{\color{black}}%
      \expandafter\def\csname LT4\endcsname{\color{black}}%
      \expandafter\def\csname LT5\endcsname{\color{black}}%
      \expandafter\def\csname LT6\endcsname{\color{black}}%
      \expandafter\def\csname LT7\endcsname{\color{black}}%
      \expandafter\def\csname LT8\endcsname{\color{black}}%
    \fi
  \fi
    \setlength{\unitlength}{0.0500bp}%
    \ifx\gptboxheight\undefined%
      \newlength{\gptboxheight}%
      \newlength{\gptboxwidth}%
      \newsavebox{\gptboxtext}%
    \fi%
    \setlength{\fboxrule}{0.5pt}%
    \setlength{\fboxsep}{1pt}%
    \definecolor{tbcol}{rgb}{1,1,1}%
\begin{picture}(9070.00,9070.00)%
    \gplgaddtomacro\gplbacktext{%
      \csname LTb\endcsname%%
      \put(814,1466){\makebox(0,0)[r]{\strut{}$-40$}}%
      \put(814,2239){\makebox(0,0)[r]{\strut{}$-30$}}%
      \put(814,3011){\makebox(0,0)[r]{\strut{}$-20$}}%
      \put(814,3784){\makebox(0,0)[r]{\strut{}$-10$}}%
      \put(814,4557){\makebox(0,0)[r]{\strut{}$0$}}%
      \put(814,5329){\makebox(0,0)[r]{\strut{}$10$}}%
      \put(814,6102){\makebox(0,0)[r]{\strut{}$20$}}%
      \put(814,6874){\makebox(0,0)[r]{\strut{}$30$}}%
      \put(814,7647){\makebox(0,0)[r]{\strut{}$40$}}%
      \put(946,4274){\makebox(0,0){\strut{} }}%
      \put(2491,4274){\makebox(0,0){\strut{}$20$}}%
      \put(4037,4274){\makebox(0,0){\strut{}$40$}}%
      \put(5582,4274){\makebox(0,0){\strut{}$60$}}%
      \put(7128,4274){\makebox(0,0){\strut{}$80$}}%
      \put(8673,4274){\makebox(0,0){\strut{}$100$}}%
    }%
    \gplgaddtomacro\gplfronttext{%
      \csname LTb\endcsname%%
      \put(209,4556){\rotatebox{-270}{\makebox(0,0){\strut{}$\tau$ (kPa)}}}%
      \put(8505,4656){\makebox(0,0){\strut{}$\sigma_n$ (kPa)}}%
      \csname LTb\endcsname%%
      \put(5698,2079){\makebox(0,0)[r]{\strut{}$I = 10^{-3}: \varphi = 18.99^{\circ}, c = 0.09$ kPa}}%
      \csname LTb\endcsname%%
      \put(5698,1859){\makebox(0,0)[r]{\strut{}$I = 10^{-2}: \varphi = 20.60^{\circ}, c = -0.15$ kPa}}%
      \csname LTb\endcsname%%
      \put(5698,1639){\makebox(0,0)[r]{\strut{}$I = 10^{-1}: \varphi = 30.64^{\circ}, c = -2.16$ kPa}}%
    }%
    \gplbacktext
    \put(0,0){\includegraphics[width={453.50bp},height={453.50bp}]{./Cercle_3000_residuel}}%
    \gplfronttext
  \end{picture}%
\endgroup
}
        \caption{Déformation volumique}
        \label{fig:3000_kinetic_ev}
    \end{subfigure}
    \caption{Influence des termes cinétiques pour N=3000}
    \label{fig:3000_kinetic_comp}
\end{figure}
L'écart de type.

\begin{table}[htbp]
\centering
\small
\begin{tabular}{@{}lllllll@{}}
\toprule
\textbf{I} & $\boldsymbol{\mu}$ & $\boldsymbol{s_{\mu}}$(\%) & $\boldsymbol{\Phi}$ & $\boldsymbol{s_{\Phi}}$(\%) & $\boldsymbol{e}$ & $\boldsymbol{s_{e}}$(\%) \\
\midrule
$10^{-3}$ & 0.338 & 2.367 & 0.595 & 0.168 & 0.680 & 0.294 \\
$10^{-2}$ & 0.360 & 4.444 & 0.590 & 0.169 & 0.695 & 0.288 \\
$2 \times 10^{-2}$ & 0.406 & 4.926 & 0.583 & 0.172 & 0.714 & 0.280 \\
$4 \times 10^{-2}$ & 0.444 & 4.054 & 0.572 & 0.175 & 0.748 & 0.401 \\
$8 \times 10^{-2}$ & 0.504 & 4.365 & 0.556 & 0.180 & 0.799 & 0.375 \\
$10^{-1}$ & 0.530 & 5.283 & 0.547 & 0.366 & 0.830 & 0.843 \\
\bottomrule
\end{tabular}
\caption{Valeurs moyennes et écarts-types $\boldsymbol{s}$ de $\mu$, $\Phi$ et $e$ en fonction du nombre d'inertie pour N=3000}
\label{table_rheologie_stats}
\end{table}

\begin{figure*}[p]
    \centering
    \begin{subfigure}[b]{0.3\textwidth}
        \centering
        \scalebox{0.5}{\input{figures/3000_mu_I_fit.tex}}
        \caption{$\mu(I)$}
        \label{fig:3000_mu_I_fit}
    \end{subfigure}
    \hfill
    \begin{subfigure}[b]{0.3\textwidth}
        \centering
        \scalebox{0.5}{% GNUPLOT: LaTeX picture with Postscript
\begingroup
  \makeatletter
  \providecommand\color[2][]{%
    \GenericError{(gnuplot) \space\space\space\@spaces}{%
      Package color not loaded in conjunction with
      terminal option `colourtext'%
    }{See the gnuplot documentation for explanation.%
    }{Either use 'blacktext' in gnuplot or load the package
      color.sty in LaTeX.}%
    \renewcommand\color[2][]{}%
  }%
  \providecommand\includegraphics[2][]{%
    \GenericError{(gnuplot) \space\space\space\@spaces}{%
      Package graphicx or graphics not loaded%
    }{See the gnuplot documentation for explanation.%
    }{The gnuplot epslatex terminal needs graphicx.sty or graphics.sty.}%
    \renewcommand\includegraphics[2][]{}%
  }%
  \providecommand\rotatebox[2]{#2}%
  \@ifundefined{ifGPcolor}{%
    \newif\ifGPcolor
    \GPcolortrue
  }{}%
  \@ifundefined{ifGPblacktext}{%
    \newif\ifGPblacktext
    \GPblacktextfalse
  }{}%
  % define a \g@addto@macro without @ in the name:
  \let\gplgaddtomacro\g@addto@macro
  % define empty templates for all commands taking text:
  \gdef\gplbacktext{}%
  \gdef\gplfronttext{}%
  \makeatother
  \ifGPblacktext
    % no textcolor at all
    \def\colorrgb#1{}%
    \def\colorgray#1{}%
  \else
    % gray or color?
    \ifGPcolor
      \def\colorrgb#1{\color[rgb]{#1}}%
      \def\colorgray#1{\color[gray]{#1}}%
      \expandafter\def\csname LTw\endcsname{\color{white}}%
      \expandafter\def\csname LTb\endcsname{\color{black}}%
      \expandafter\def\csname LTa\endcsname{\color{black}}%
      \expandafter\def\csname LT0\endcsname{\color[rgb]{1,0,0}}%
      \expandafter\def\csname LT1\endcsname{\color[rgb]{0,1,0}}%
      \expandafter\def\csname LT2\endcsname{\color[rgb]{0,0,1}}%
      \expandafter\def\csname LT3\endcsname{\color[rgb]{1,0,1}}%
      \expandafter\def\csname LT4\endcsname{\color[rgb]{0,1,1}}%
      \expandafter\def\csname LT5\endcsname{\color[rgb]{1,1,0}}%
      \expandafter\def\csname LT6\endcsname{\color[rgb]{0,0,0}}%
      \expandafter\def\csname LT7\endcsname{\color[rgb]{1,0.3,0}}%
      \expandafter\def\csname LT8\endcsname{\color[rgb]{0.5,0.5,0.5}}%
    \else
      % gray
      \def\colorrgb#1{\color{black}}%
      \def\colorgray#1{\color[gray]{#1}}%
      \expandafter\def\csname LTw\endcsname{\color{white}}%
      \expandafter\def\csname LTb\endcsname{\color{black}}%
      \expandafter\def\csname LTa\endcsname{\color{black}}%
      \expandafter\def\csname LT0\endcsname{\color{black}}%
      \expandafter\def\csname LT1\endcsname{\color{black}}%
      \expandafter\def\csname LT2\endcsname{\color{black}}%
      \expandafter\def\csname LT3\endcsname{\color{black}}%
      \expandafter\def\csname LT4\endcsname{\color{black}}%
      \expandafter\def\csname LT5\endcsname{\color{black}}%
      \expandafter\def\csname LT6\endcsname{\color{black}}%
      \expandafter\def\csname LT7\endcsname{\color{black}}%
      \expandafter\def\csname LT8\endcsname{\color{black}}%
    \fi
  \fi
    \setlength{\unitlength}{0.0500bp}%
    \ifx\gptboxheight\undefined%
      \newlength{\gptboxheight}%
      \newlength{\gptboxwidth}%
      \newsavebox{\gptboxtext}%
    \fi%
    \setlength{\fboxrule}{0.5pt}%
    \setlength{\fboxsep}{1pt}%
    \definecolor{tbcol}{rgb}{1,1,1}%
\begin{picture}(7200.00,5040.00)%
    \gplgaddtomacro\gplbacktext{%
      \csname LTb\endcsname%%
      \put(946,704){\makebox(0,0)[r]{\strut{}$0.53$}}%
      \csname LTb\endcsname%%
      \put(946,1218){\makebox(0,0)[r]{\strut{}$0.54$}}%
      \csname LTb\endcsname%%
      \put(946,1733){\makebox(0,0)[r]{\strut{}$0.55$}}%
      \csname LTb\endcsname%%
      \put(946,2247){\makebox(0,0)[r]{\strut{}$0.56$}}%
      \csname LTb\endcsname%%
      \put(946,2762){\makebox(0,0)[r]{\strut{}$0.57$}}%
      \csname LTb\endcsname%%
      \put(946,3276){\makebox(0,0)[r]{\strut{}$0.58$}}%
      \csname LTb\endcsname%%
      \put(946,3790){\makebox(0,0)[r]{\strut{}$0.59$}}%
      \csname LTb\endcsname%%
      \put(946,4305){\makebox(0,0)[r]{\strut{}$0.6$}}%
      \csname LTb\endcsname%%
      \put(946,4819){\makebox(0,0)[r]{\strut{}$0.61$}}%
      \csname LTb\endcsname%%
      \put(1253,484){\makebox(0,0){\strut{}$10^{-4}$}}%
      \csname LTb\endcsname%%
      \put(3055,484){\makebox(0,0){\strut{}$10^{-3}$}}%
      \csname LTb\endcsname%%
      \put(4858,484){\makebox(0,0){\strut{}$10^{-2}$}}%
      \csname LTb\endcsname%%
      \put(6660,484){\makebox(0,0){\strut{}$10^{-1}$}}%
      \put(1651,1733){\makebox(0,0)[l]{\strut{}$\Phi_{max} = 0.5955$}}%
      \put(1651,1321){\makebox(0,0)[l]{\strut{}$b = 0.5193$}}%
    }%
    \gplgaddtomacro\gplfronttext{%
      \csname LTb\endcsname%%
      \put(209,2761){\rotatebox{-270}{\makebox(0,0){\strut{}$\phi$}}}%
      \put(3940,154){\makebox(0,0){\strut{}$I$}}%
      \csname LTb\endcsname%%
      \put(3322,1097){\makebox(0,0)[r]{\strut{}Données}}%
      \csname LTb\endcsname%%
      \put(3322,877){\makebox(0,0)[r]{\strut{}$\Phi(I)$ régression}}%
    }%
    \gplbacktext
    \put(0,0){\includegraphics[width={360.00bp},height={252.00bp}]{./3000_phi_I_fit}}%
    \gplfronttext
  \end{picture}%
\endgroup
}
        \caption{$\Phi(I)$}
        \label{fig:3000_phi_I_fit}
    \end{subfigure}
    \hfill
    \begin{subfigure}[b]{0.3\textwidth}
        \centering
        \scalebox{0.5}{% GNUPLOT: LaTeX picture with Postscript
\begingroup
  \makeatletter
  \providecommand\color[2][]{%
    \GenericError{(gnuplot) \space\space\space\@spaces}{%
      Package color not loaded in conjunction with
      terminal option `colourtext'%
    }{See the gnuplot documentation for explanation.%
    }{Either use 'blacktext' in gnuplot or load the package
      color.sty in LaTeX.}%
    \renewcommand\color[2][]{}%
  }%
  \providecommand\includegraphics[2][]{%
    \GenericError{(gnuplot) \space\space\space\@spaces}{%
      Package graphicx or graphics not loaded%
    }{See the gnuplot documentation for explanation.%
    }{The gnuplot epslatex terminal needs graphicx.sty or graphics.sty.}%
    \renewcommand\includegraphics[2][]{}%
  }%
  \providecommand\rotatebox[2]{#2}%
  \@ifundefined{ifGPcolor}{%
    \newif\ifGPcolor
    \GPcolortrue
  }{}%
  \@ifundefined{ifGPblacktext}{%
    \newif\ifGPblacktext
    \GPblacktextfalse
  }{}%
  % define a \g@addto@macro without @ in the name:
  \let\gplgaddtomacro\g@addto@macro
  % define empty templates for all commands taking text:
  \gdef\gplbacktext{}%
  \gdef\gplfronttext{}%
  \makeatother
  \ifGPblacktext
    % no textcolor at all
    \def\colorrgb#1{}%
    \def\colorgray#1{}%
  \else
    % gray or color?
    \ifGPcolor
      \def\colorrgb#1{\color[rgb]{#1}}%
      \def\colorgray#1{\color[gray]{#1}}%
      \expandafter\def\csname LTw\endcsname{\color{white}}%
      \expandafter\def\csname LTb\endcsname{\color{black}}%
      \expandafter\def\csname LTa\endcsname{\color{black}}%
      \expandafter\def\csname LT0\endcsname{\color[rgb]{1,0,0}}%
      \expandafter\def\csname LT1\endcsname{\color[rgb]{0,1,0}}%
      \expandafter\def\csname LT2\endcsname{\color[rgb]{0,0,1}}%
      \expandafter\def\csname LT3\endcsname{\color[rgb]{1,0,1}}%
      \expandafter\def\csname LT4\endcsname{\color[rgb]{0,1,1}}%
      \expandafter\def\csname LT5\endcsname{\color[rgb]{1,1,0}}%
      \expandafter\def\csname LT6\endcsname{\color[rgb]{0,0,0}}%
      \expandafter\def\csname LT7\endcsname{\color[rgb]{1,0.3,0}}%
      \expandafter\def\csname LT8\endcsname{\color[rgb]{0.5,0.5,0.5}}%
    \else
      % gray
      \def\colorrgb#1{\color{black}}%
      \def\colorgray#1{\color[gray]{#1}}%
      \expandafter\def\csname LTw\endcsname{\color{white}}%
      \expandafter\def\csname LTb\endcsname{\color{black}}%
      \expandafter\def\csname LTa\endcsname{\color{black}}%
      \expandafter\def\csname LT0\endcsname{\color{black}}%
      \expandafter\def\csname LT1\endcsname{\color{black}}%
      \expandafter\def\csname LT2\endcsname{\color{black}}%
      \expandafter\def\csname LT3\endcsname{\color{black}}%
      \expandafter\def\csname LT4\endcsname{\color{black}}%
      \expandafter\def\csname LT5\endcsname{\color{black}}%
      \expandafter\def\csname LT6\endcsname{\color{black}}%
      \expandafter\def\csname LT7\endcsname{\color{black}}%
      \expandafter\def\csname LT8\endcsname{\color{black}}%
    \fi
  \fi
    \setlength{\unitlength}{0.0500bp}%
    \ifx\gptboxheight\undefined%
      \newlength{\gptboxheight}%
      \newlength{\gptboxwidth}%
      \newsavebox{\gptboxtext}%
    \fi%
    \setlength{\fboxrule}{0.5pt}%
    \setlength{\fboxsep}{1pt}%
    \definecolor{tbcol}{rgb}{1,1,1}%
\begin{picture}(7200.00,5040.00)%
    \gplgaddtomacro\gplbacktext{%
      \csname LTb\endcsname%%
      \put(946,704){\makebox(0,0)[r]{\strut{}$0.66$}}%
      \csname LTb\endcsname%%
      \put(946,1078){\makebox(0,0)[r]{\strut{}$0.68$}}%
      \csname LTb\endcsname%%
      \put(946,1452){\makebox(0,0)[r]{\strut{}$0.7$}}%
      \csname LTb\endcsname%%
      \put(946,1826){\makebox(0,0)[r]{\strut{}$0.72$}}%
      \csname LTb\endcsname%%
      \put(946,2200){\makebox(0,0)[r]{\strut{}$0.74$}}%
      \csname LTb\endcsname%%
      \put(946,2574){\makebox(0,0)[r]{\strut{}$0.76$}}%
      \csname LTb\endcsname%%
      \put(946,2949){\makebox(0,0)[r]{\strut{}$0.78$}}%
      \csname LTb\endcsname%%
      \put(946,3323){\makebox(0,0)[r]{\strut{}$0.8$}}%
      \csname LTb\endcsname%%
      \put(946,3697){\makebox(0,0)[r]{\strut{}$0.82$}}%
      \csname LTb\endcsname%%
      \put(946,4071){\makebox(0,0)[r]{\strut{}$0.84$}}%
      \csname LTb\endcsname%%
      \put(946,4445){\makebox(0,0)[r]{\strut{}$0.86$}}%
      \csname LTb\endcsname%%
      \put(946,4819){\makebox(0,0)[r]{\strut{}$0.88$}}%
      \csname LTb\endcsname%%
      \put(1333,484){\makebox(0,0){\strut{}$10^{-3}$}}%
      \csname LTb\endcsname%%
      \put(3964,484){\makebox(0,0){\strut{}$10^{-2}$}}%
      \csname LTb\endcsname%%
      \put(6595,484){\makebox(0,0){\strut{}$10^{-1}$}}%
    }%
    \gplgaddtomacro\gplfronttext{%
      \csname LTb\endcsname%%
      \put(209,2761){\rotatebox{-270}{\makebox(0,0){\strut{}$e$}}}%
      \put(3940,154){\makebox(0,0){\strut{}$I$}}%
      \csname LTb\endcsname%%
      \put(3322,4646){\makebox(0,0)[r]{\strut{}Données}}%
      \csname LTb\endcsname%%
      \put(3322,4426){\makebox(0,0)[r]{\strut{}e(I) régression}}%
    }%
    \gplbacktext
    \put(0,0){\includegraphics[width={360.00bp},height={252.00bp}]{./3000_e_I_plot}}%
    \gplfronttext
  \end{picture}%
\endgroup
}
        \caption{$e(I)$}
        \label{fig:3000_e_I_fit}
    \end{subfigure}
    \caption{Rhéologies $\mu(I)$, $\Phi(I)$ et $e(I)$ quand $\epsilon_{yy} = 40 \div 60\%$ pour N=3000}
    \label{fig:rheologies_3000}
\end{figure*}

\begin{figure*}[p]
    \centering
    \begin{subfigure}[b]{0.3\textwidth}
        \centering
        \scalebox{0.5}{\input{figures/1000_mu_I_fit.tex}}
        \caption{$\mu(I)$}
        \label{fig:1000_mu_I_fit}
    \end{subfigure}
    \hfill
    \begin{subfigure}[b]{0.3\textwidth}
        \centering
        \scalebox{0.5}{\input{figures/1000_phi_I_fit.tex}}
        \caption{$\Phi(I)$}
        \label{fig:1000_phi_I_fit}
    \end{subfigure}
    \hfill
    \begin{subfigure}[b]{0.3\textwidth}
        \centering
        \scalebox{0.5}{\input{figures/1000_e_I_fit.tex}}
        \caption{$e(I)$}
        \label{fig:1000_e_I_fit}
    \end{subfigure}
    \caption{Rhéologies $\mu(I)$, $\Phi(I)$ et $e(I)$ quand $\epsilon_{yy} = 40 \div 60\%$ pour N=1000}
    \label{fig:rheologies_1000}
\end{figure*}

\section{Conclusion}\label{discussion}

% \begin{gather}
% \sum_{i=0}^{m}x_{ijk} = y_{jk}\qquad \forall j\in \{0\} \cup J, k \in K \label{sommet_atteignable_a_un_arc_entrant}\\ 
% \sum_{i=0}^{n+1+m}x_{jik} = y_{jk}\qquad \forall j\in J \cup \{m+1\}, k \in K \label{sommet_atteignable_a_un_arc_sortant}\\ 
% \sum_{j=0}^{m}x_{0jk} = 1 \qquad \forall k \in K \label{tournee_commence_depot}\\ 
% \sum_{j=1}^{n+1+m}x_{j0k} = 1 \qquad \forall k \in K \label{tournee_termine_depot}\\
% \sum_{j=1}^{m}\sum_{k=1}^{l} \alpha_{ij}D_{ijk} \geqslant d_{i}\qquad \forall i\in I \label{demande_satisfaite}\\
% \sum_{i=1}^{n}D_{ijk} \leqslant Q_{k}y_{jk}\qquad \forall k \in K, j\in J \label{distribution_sommet_atteignable}\\
% \sum_{i=1}^{n}\sum_{j=1}^{m}D_{ijk} \leqslant Q_k\qquad \forall k \in K \label{capacite_vehicule}\\
% u_{ik} + c_{ij} - (1 - x_{ijk})T \leqslant u{jk} \notag \\
% \qquad \forall i \in \{0\} \cup J, j \in J \cup \{0,n+1+m\}, k \in K \label{sub_tour}\\
% x_{ijk} \in\left\{0,1\right\} \notag \\
% \qquad\forall i,j\in\{0\}\cup J \cup \{n+m+1\}, i \neq j, k \in K \\
% y_{jk} \in\left\{0,1\right\} \forall j\in J, k \in K\\ 
% u_{ik} \geqslant 0  \qquad\forall i\in J, k \in K\\ 
% D_{ijk} \geqslant 0 \qquad\forall i\in I, j\in J, k \in K\\ 
% \end{gather}



\bibliography{bibliographie.bib}

\end{document}