%!TEX root = main.tex
\chapter{Contexte}

Ce projet de thèse porte sur le développement d'une modélisation mécanique numérique pour l’écoulement de matériaux complexes. L'aspect novateur de cette approche réside dans la capacité à intégrer, au sein d'une même simulation, deux échelles significativement différentes, telles que celles associées à des particules de sol et celle d’une géostructure. La méthodologie numérique adoptée pour cette étude repose sur une approche à double échelle, où la gestion de la grande échelle est assurée par la méthode lagrangienne des points matériels (MPM), permettant de traiter des transformations finies à grande échelle. La petite échelle est introduite grâce à des lois constitutives élaborées en cours de simulation par homogénéisation numérique. Il s’agit ici de simulations en éléments discrets (DEM) sophistiquées. Ces approches à double échelle, dites MPM$\times$DEM, sont novatrices par leur concept, mais aussi du fait que la MPM est une méthode relativement récente, et demande encore à être améliorée. Le projet propose d’étudier une diversité de matériaux et de problématiques avec, en ligne de mire, certains enjeux environnementaux. Spécifiquement, la protection face à des risques naturels (avalanches rocheuses, de coulées de boues, ou tout autre mouvement de terrain…) qui deviennent de plus en plus manifestes de nos jours à cause des changements climatiques. 


La modélisation à double échelle dont il est question ici instaure une pratique moderne, où la petite échelle constitue le modèle constitutif sans définir de limite spatiale précise des volumes concernés ; Feyel (1999). L'idée de modéliser la réponse mécanique d'une microstructure -- éventuellement couplée à d'autres phénomènes physiques comme les processus thermomécaniques ou hydromécaniques -- en vue de transposer les résultats de cette analyse microscopique à l'échelle macroscopique, n’est pas tout à fait nouvelle. Les méthodes d'homogénéisation ont été développées dans cet esprit depuis des décennies, conduisant à des travaux théoriques remarquables, qui, en général, aboutissent à des informations précieuses sur la forme que devrait avoir une équation constitutive macroscopique pour représenter fidèlement le milieu considéré. Dans l'approche classique, l'analyse à microéchelle, qu'elle soit théorique ou numérique, est effectuée séparément et indépendamment d'une éventuelle analyse à macro-échelle utilisant le modèle homogénéisé. Sans passer en revue le vaste domaine des méthodes d'homogénéisation, le projet de thèse se concentrera plutôt sur une méthode d'homogénéisation numérique à double échelle, impliquant la MPM à grande échelle, et la DEM à la petite échelle. La démarche peut être décrite comme un double processus, qui implique simultanément la résolution d'un problème à la microéchelle, jouant le rôle d'une méthode d'homogénéisation numérique, et la résolution du problème aux limites à la macro-échelle représentant la géostructure à analyser. Les deux processus échangent des informations tout au long de la résolution du problème. Le point clé de cette méthode réside dans le fait que le calcul DEM à microéchelle joue le rôle de la Loi Constitutive Homogénéisée Numériquement (LCHN) pour le calcul macroscopique. Étant donné une déformation macroscopique, il renvoie la contrainte macroscopique correspondante comme illustrée sur la Figure~\ref{fig:2scales}. Elle est utilisée exactement comme toute autre loi constitutive macroscopique « classique », et c’est précisément sur ce point que le paradigme de LCHN diffère de la conception première des FE2 (e.g., Feyel, 1999). 


Un outil de calcul MPM a été développé au Laboratoire 3SR. Il s’agit d’une approche dynamique permettant la simulation de très grandes déformations et bien adaptée aux écoulements gravitaires. L'implémentation d’un couplage multiéchelle avec la méthode DEM a également déjà été effectuée. L'enjeu, jusqu’ici, est de remplacer les modèles constitutifs analytiques par des simulations DEM qui jouent alors le rôle de loi de comportement, mais qui sont infiniment plus riches. Certains aspects propres à l’approche MPM, lorsqu’elle est employée avec des LCHN, restent malgré tout encore améliorables. Parmi elles, ce projet de thèse a l’ambition de traiter spécifiquement deux verrous – ils sont détaillés ci-après.
 
\begin{figure}[ht]
  \centering
  \includegraphics[width=1.0\columnwidth]{figs/2scales-scheme.png}
  \caption{Principe d’une approche à deux échelles simultanées (ou hiérarchiques) telle qu'elle est développée au Laboratoire 3SR. La partie à gauche représente un système MPM où on peut voir les nœuds (bleus) de grille fixe, les points matériels (rouges). La partie à droite montre un volume périodique DEM se déformant sous la demande d’un gradient de vitesse de la part d’un point matériel, et répondant par une contrainte obtenue par homogénéisation numérique. \label{fig:2scales}}
\end{figure}


\textbf{Verrou 1} : équivalence des inerties aux deux échelles. Diverses nouvelles interrogations théoriques se concentrent sur l'équivalence des inerties aux deux échelles. Cette nouvelle problématique émerge de l'utilisation de la MPM, en remplacement des modèles statiques, couramment employés. La méthode conventionnelle consiste à censurer les réponses mécaniques dynamiques de la DEM en attendant une stabilisation statique systématique (e.g., Nguyen et al., 2014; Guo \& Zhao, 2014 ; Weijian \& Zhao, 2019). Cela nous prive de la capacité à modéliser des phénomènes très dynamiques, tels que ceux observés dans le déclenchement de mouvement de terrain catastrophique.  

\textbf{Verrou 2} : conditions aux limites et initiales. À l'heure actuelle, il est important de souligner que la prise en compte des conditions aux limites dans notre outil de calcul MPM n'est pas réalisée de manière pleinement satisfaisante. Cette lacune représente un obstacle significatif, et l'un des défis à relever consistera à mettre en œuvre des solutions existantes (e.g., Nakamura et al., 2021) de manière précise et, si nécessaire, à les adapter pour assurer une modélisation plus robuste des conditions aux limites. Ce processus d'amélioration permettra d'ajouter une dimension de précision et de fiabilité à notre outil, renforçant ainsi sa capacité à fournir des résultats plus représentatifs et conformes aux réalités physiques des systèmes étudiés. L'intégration rigoureuse des conditions aux limites constituera donc une étape cruciale pour augmenter la crédibilité et l'applicabilité de notre méthode de calcul MPM dans divers contextes.

\begin{figure}[ht]
  \centering
  \includegraphics[width=1.0\columnwidth]{figs/simu-biplan.png}
  \caption{Simulation par éléments discrets de l'écoulement d'une masse initialement instable sur une surface inclinée. Le mouvement se stabilise sur une surface plane en traversant une zone de transition où divers mécanismes de dissipation entrent en jeu.\label{fig:biplan}}
\end{figure}



Surmonter ces verrous, que ce soit intégralement ou partiellement, ouvrira la voie à la résolution de divers problèmes. L'avantage de l'approche à double échelle réside dans sa capacité à simuler des problèmes à grande échelle tout en préservant l'accès aux caractéristiques de la petite échelle. Cela offre des opportunités pour mener diverses analyses micromécaniques. On pourra ainsi examiner les effets de la forme des particules, leur propension à la rupture, la résistance des liens qui les unissent, ainsi que leurs interactions avec un fluide, le tout à l'échelle de systèmes mécaniques de grande envergure comme celui de la Figure~\ref{fig:biplan}.

Ce projet de thèse innove en proposant une approche d'homogénéisation numérique à double échelle pour la modélisation des mouvements gravitaires en lien avec les changements climatiques. Deux verrous spécifiques, liés à l'équivalence des inerties et à la prise en compte rigoureuse des conditions aux limites, seront abordés pour renforcer l'applicabilité de la méthode. Des simulations préliminaires démontrent la pertinence de l'approche dans des scénarios complexes. Le travail sera encadré par une équipe expérimentée du Laboratoire 3SR (Vincent RICHEFEU, MCF HDR et Gaël COMBE, PR), en collaboration avec l’École de Génie Civil d’Hanoï (Trung Kien NGUYEN, MCF). Les 3 personnes ont une coopération de longue date notamment sur les approches multi-échelles qui ont donné lieu à plusieurs publications ensemble (Nguyen et al., 2017; Desrues et al.,  2019; Nguyen et al., 2022). Ce projet ouvre des perspectives pour des analyses micromécaniques approfondies, offrant des solutions aux problèmes géotechniques complexes et contribueront à une meilleure compréhension des risques naturels. De plus, le Vietnam, un pays tropical, a été fortement touché par le changement climatique et les événements météorologiques extrêmes, en particulier dans les régions du nord, du centre et des hautes terres, entraînant des glissements de terrain et des coulées de débris catastrophiques. Cette thèse pourrait grandement contribuer à une meilleure compréhension des mécanismes de déclenchement et de propagation de ces événements et conduire à une meilleure stratégie de gestion des risques naturels.



